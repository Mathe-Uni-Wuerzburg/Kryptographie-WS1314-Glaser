\begin{section}{Übung 2}
 \begin{subsection}{Aufgabe 1}
  \begin{enumerate}
   \item $(\{0,1,2,3,4\}, f)$ mit $f(x,y) = (x+y) \mod  5$ ist eine endliche kommutative Gruppe
   \begin{itemize}
    \item $f: \mathbb{Z}_5 \times \mathbb{Z}_5 \rightarrow \mathbb{Z}_5$ ist total \checkmark
    \item Assoziativität: 
    \begin{align*}
      f(f(a,b),c) & = f((a+b) \mod  5,c) \\
      & = ((a+b) \mod 5 +c) \mod 5 \\
      & = (a+b+c) \mod 5 = (a+(b+c)\mod 5)\mod 5 \\
      & = f(a,f(b,c)) \checkmark
    \end{align*}
    \item Neutrales Element: $f(x,0) = (x+0)\mod 5 = x$ \\
    $f(0,x) = (0+x)\mod 5 = x$
    \item Inverses Element: Sei $a \in \mathbb{Z}_5$:
    \begin{align*}
      b &= (-a)\mod 5 \\
      &= (a+(-a)\mod 5)\mod 5 \\
      &= 0\mod 5 = 0
    \end{align*}
    Eindeutigkeit: $0\mod 5 =  0$ \checkmark \\
    Angenommen $f(a,b') = 0  = f(a,b)$ mit $b' \in \mathbb{Z}_5$ \\
    $\Rightarrow b' \mod 5 = b\mod 5$\\
    $\Rightarrow b' = b$\\
    $\Rightarrow$ genau ein inverses Element.
    \item Kommutativität: \\
    $f(a,b) = f(b,a) = (a+b)\mod 5 = (b+a)\mod 5 = f(b,a) \checkmark$
   \end{itemize}
    \item $(\mathbb{Z}_6,f,g)$ mit $f(x,y) = (x+y)\mod 6$, $g(x,y) = (x\cdot y)\mod 6$ ist kein Körper: \\
    Damit $(\mathbb{Z}_6),f,g)$ ein Körper ist, muss $(\{1,2,3,4,5\},g)$ kommutative Gruppe sein. \\
    $\Rightarrow \forall a \in \{1,2,3,4,5\} \exists! b \in {1,2,3,4,5}: [g(a,b) = 1]$
    \begin{itemize}
     \item $g(2,1) = 2$
     \item $g(2,2) = 4$
     \item $g(2,3) = 0$
     \item $g(2,4) = 2$
     \item $g(2,5) = 4$
    \end{itemize}
    $\Rightarrow$ 2 besitzt kein inverses Element in $(\{1,2,3,4,5\},g)$ \\
    $\Rightarrow (\mathbb{Z}_6,f,g)$ ist kein Körper.

  \end{enumerate}

 \end{subsection}
 \begin{subsection}{Hinweise zu Übungsblatt 3}
  Monoalphabetische Verschlüsselung (Substitutionschiffre): \\
  ABCDEFGHJKLMNOPQRSTUVWXYZ\\ pfeile und so\\ \\ \\
  $S = (\Sigma, \mathcal{K}, \mathcal{E}, \mathcal{D})$ 
  \begin{itemize}
   \item $\Sigma = \{A,\cdots,Z\}$
   \item Schlüssel: $\pi : \Sigma \rightarrow \Sigma$ Bijektion
   \begin{align*}
    P &= \text{ Menge aller Permutationen auf } \Sigma \\
    &= \{\pi: \Sigma \rightarrow \Sigma | \pi \text{ bijektiv}\}
   \end{align*}
   \item $\mathcal{K}(1^n)$ liefert gleichverteilt Element aus $\{(\pi,\pi)|\pi \in P\}$
   \item $\mathcal{E}(\pi,m_1,m_2, \cdots, m_n) = \pi(m_1) \pi(m_2) \cdots \pi(m_n)$
   \item $\mathcal{D} = \pi^{-1}(c_1) \pi^{-1}(c_2) \cdots \pi^{-1}(c_n)$
  \end{itemize}
  
  Ist $S$ perfekt sicher? \\
  \begin{itemize}
   \item Betrachte Klartext der Länge 1 \\
   $P_{\Sigma^1}(a) = \frac{1}{26}$ für alle $a \in \Sigma$\\
   Sei nun $m \in \Sigma^1$ beliebig. \\
   $P(E_m) = \frac{1}{26}$ [Wahrscheinlichkeit, das Klartext m gewählt wird.] \\
   Sei $c \in C_1$ beliebig. [$C_1 = \Sigma$ alle möglichen Chiffretexte der Länge 1]
   \begin{align*}
      P(E_m|E_c) &= P(\mathcal{K}(1) \text{ liefert Permutation }\pi\text{ mit } \pi(m) = c)\\
      &= \frac{25!}{26!} = \frac{1}{26} = P(E_m)
   \end{align*}
   $\Rightarrow$ S ist perfekt sicher bezüglich $P_{\Sigma^1}$
   \item Betrachte gleichverteilte Klartexte der Länge 2 \\
   $P_{\Sigma^2}(m) = \frac{1}{26^2}$ für alle $m \in \Sigma^2$ [$|\Sigma^2| = 26^2$]\\
   Wähle $m=AF \in \Sigma^2$: \\
   $P(E_m) = \frac{1}{26^2}$ \\
   $C_2 = \Sigma^2$ \\
   Sei $c = ww \in C_2$
   \begin{align*}
    P(E_m|E_c) &= P(\mathcal{K}(11) \text{ liefert Permutation }\pi\text{ mit }\pi(A)=w\text{ und }\pi(F)=w)\\
    &= 0 \neq P(E_m) 
   \end{align*}

   
   
   $\Rightarrow$ S nicht perfekt sicher bezüglich $P_{\Sigma^2}$\\
   $\Rightarrow$ S nicht perfekt sicher

   
   
  \end{itemize}


  
  
  
 \end{subsection}


\end{section}
