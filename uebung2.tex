\begin{section}{Übung 2}
 \begin{subsection}{Aufgabe 1}
  \begin{enumerate}
   \item $(\{0,1,2,3,4\}, f)$ mit $f(x,y) = (x+y) \mod  5$ ist eine endliche kommutative Gruppe
   \begin{itemize}
    \item $f: \mathbb{Z}_5 \times \mathbb{Z}_5 \rightarrow \mathbb{Z}_5$ ist total \checkmark
    \item Assoziativität: 
    \begin{align*}
      f(f(a,b),c) & = f((a+b) \mod  5,c) \\
      & = ((a+b) \mod 5 +c) \mod 5 \\
      & = (a+b+c) \mod 5 = (a+(b+c)\mod 5)\mod 5 \\
      & = f(a,f(b,c)) \checkmark
    \end{align*}
    \item Neutrales Element: $f(x,0) = (x+0)\mod 5 = x$ \\
    $f(0,x) = (0+x)\mod 5 = x$
    \item Inverses Element: Sei $a \in \mathbb{Z}_5$:
    \begin{align*}
      b &= (-a)\mod 5 \\
      &= (a+(-a)\mod 5)\mod 5 \\
      &= 0\mod 5 = 0
    \end{align*}
    Eindeutigkeit: $0\mod 5 =  0$ \checkmark \\
    Angenommen $f(a,b') = 0  = f(a,b)$ mit $b' \in \mathbb{Z}_5$ \\
    $\Rightarrow b' \mod 5 = b\mod 5$\\
    $\Rightarrow b' = b$\\
    $\Rightarrow$ genau ein inverses Element.
    \item Kommutativität: \\
    $f(a,b) = f(b,a) = (a+b)\mod 5 = (b+a)\mod 5 = f(b,a) \checkmark$
   \end{itemize}
    \item $(\mathbb{Z}_6,f,g)$ mit $f(x,y) = (x+y)\mod 6$, $g(x,y) = (x\cdot y)\mod 6$ ist kein Körper: \\
    Damit $(\mathbb{Z}_6),f,g)$ ein Körper ist, muss $(\{1,2,3,4,5\},g)$ kommutative Gruppe sein. \\
    $\Rightarrow \forall a \in \{1,2,3,4,5\} \exists! b \in {1,2,3,4,5}: [g(a,b) = 1]$
    \begin{itemize}
     \item $g(2,1) = 2$
     \item $g(2,2) = 4$
     \item $g(2,3) = 0$
     \item $g(2,4) = 2$
     \item $g(2,5) = 4$
    \end{itemize}
    $\Rightarrow$ 2 besitzt kein inverses Element in $(\{1,2,3,4,5\},g)$ \\
    $\Rightarrow (\mathbb{Z}_6,f,g)$ ist kein Körper.

  \end{enumerate}

 \end{subsection}

\end{section}
