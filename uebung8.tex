\begin{section}{8. Übung}
 \begin{subsection}{Aufgabe 1}
  $B \equiv g^b \equiv g^{p-1-x} \equiv g^{p-1}\cdot g^{-x} \equiv g^{-x} \equiv (g^{-1})^x \mod p$\\
  $x$ hat höchstens 4 Einser in Binärdarstelung. \\
  $x=2^i+2^j+2^k+2^l$\\
  $h^x = h^{2^i+2^j+2^k+2^l} = h^{2^i} \cdot h^{2^j} \cdot h^{2^k} \cdot h^{2^l}$
 \end{subsection}
 \begin{subsection}{Zusatzaufgabe}
  $\log_{g,p}$ konnte man effizient berechnen. Man will: $\log_{h,p}$ berechnen. ($p$ Prim, $g,h$ Erzeuger in $\mathbb{Z}_p^*$ )\\
  Behauptung:  $\log_{n,p} x= (\log_{g,p} x \cdot \log_{g,p} h ^{-1}) \mod (p-1)$\\
  Beweis: Zeigen, dass $\log_{g,p} h ^{-1}$ existiert, d.h. \\
  $ggT(\log_{g,p} h, p-1) = 1 (\Leftrightarrow \log_{g,p} h \in \mathbb{Z}_{p-1}^*)$\\
  wäre $ggT(\log_{g,p} h, p-1) = d > 1$. Dann gilt: $h^{\frac{p-1}{d}} \equiv (g^{\log_{g,p} h})^{\frac{p-1}{d}} \equiv \underbrace{(g^{\frac{\log_{g,p} h}{d}})^{p-1}}_{\in \mathbb{Z}_p^*} \equiv 1 \mod p \Rightarrow ord_p h \leq \frac{p-1}{d}<p-1 = ord_p h$, Widerspruch.\\
  $x \equiv y \mod p-1 \Rightarrow z^x \equiv z^y \mod p$\\
  $x = y +k(p-1) \Rightarrow z^x = z^y\cdot z^{k(p-1)} = z^y \underbrace{(z^k)^{p-1}}_{\equiv 1 \mod p} \equiv z^y \mod p $\\
  $h^{\log_{g,p} x \cdot \log_{g,p} h ^{-1} \mod p-1} \equiv h^{\log_{g,p} x \cdot \log_{g,p} h ^{-1}} \equiv g^{\overbrace{\log_{g,p} h \cdot \log_{g,p} h ^{-1}}^{\equiv 1 \mod p-1}\cdot \log_{g,p} x} \equiv g^{1\cdot \log_{g,p} x} \equiv x \mod p$
 \end{subsection}
 \begin{subsection}{Extra}
  \begin{itemize}
   \item Alice entscheidet sich für Kopf oder Zahl
   \item Bob wirft eine Münze
  \end{itemize}
  $\Rightarrow$ Gewinner steht fest.\\
  Alice wählt Primzahlen $p,q$ mit $p\equiv q \equiv 3 \mod 4$\\
  $Rightarrow$ Alice sendet $n=p\cdot q$\\
  Alice entscheidet sich für Kopf (1) oder Zahl (0) $\rightarrow b$\\
  $\Rightarrow$ Alice sendet $c = -1^b \cdot r^2 \mod n $ für ein beliebiges $r\in \mathbb{Z}_n^*$\\
  Bob wirft die Münze $\rightarrow x \in \{0,1\}$\\
  $\Leftarrow$ Bob sendet Ergebnis $x$ \\
  Alice weiß jetzt, ob sie gewonnen hat.\\
  $\Rightarrow$ Alice schickt $p,q$\\
  Bob kann $c$ entschlüsseln und erhält $b$\\
  $\Rightarrow$ Beide kennen den Gewinner.
 \end{subsection}
\end{section}
